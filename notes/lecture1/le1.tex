\documentclass[swedish]{article}

\usepackage[a4paper,left=1in,right=1in,top=1.25in,bottom=1.25in]{geometry}
\usepackage[style=iso]{datetime2}
\usepackage{graphicx}
\usepackage{amssymb}

\author{Arvid Karlgren}
\title{Föreläsning 1\\
       \LARGE Gränsvärden: Definition och räkneregler}

\renewcommand{\contentsname}{Innehåll}
\setlength\parindent{0pt}

\begin{document}

\maketitle

\tableofcontents

\pagebreak

\section{Kursens mål}

Kursen kommer att hantera följande områden:

\begin{enumerate}
    \item Kontinuitet
    \item Gränsvärden
    \item Derivata
    \item Funktionsundersökning
    \item Primitiva funktioner
    \item Integraler
\end{enumerate}

\section{Gränsvärden}

\subsection{Definition}

Gränsvärden handlar om hur en funktion ser ut (vilka värden den antar) när x närmar sig olika värden. Det finns två typer av gränsvärden.

\begin{itemize}
    \item{Nära (men ej i) en punkt $a\in x$.}
    \item{För obegränsat stora positiva eller negativa $x\in \mathbb{R}$.}
\end{itemize}

Gränsvärden betecknas med $\to$, till exempel $x \to a$ ("$x$ går mot $a$").

Figur 1 visar några fall där den exakta definitionen av gränsvärden spelar stor roll. 

\begin{figure}[h!]
    \includegraphics[width=\linewidth]{figure1.png}
    \label{fig:figure1}
    \caption{Olika fall för gränsvärden.}
\end{figure}

Utifrån figur 1 vill vi, utifrån definitionen för gränsvärden, kunna säga följande:

\begin{itemize}
    \item{$f(x) \to A$ då $x \to x_1$, ($x_1 \notin D_f$), alternativt: \\
        \[\lim_{x \to x_1} f(x) = A\]}
    \item{$f(x) \to A$ då $x \to x_2$, ($x_2 \in D_f$, $f(x_2) = B$).}
    \item{$f(x) \to A$ då $x \to x_3$, ($x_3 \in D_f$, $f(x_3) = A$).}
    \item{$f(x)$ saknar gränsvärden då $x \to x_4$ eller $x \to x_5$.}
\end{itemize}

\subsection{Räkneregler}

\subsection{Exempel}



\end{document}
